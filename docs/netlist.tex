\documentclass{scrartcl}

\usepackage{cmap}
\usepackage[T1]{fontenc}
\usepackage{lmodern}
\usepackage[french]{babel}
\usepackage{hyperref}

\title{Simulateur de Netlist}
\author{Adrien \textsc{Mathieu}}
\date{mercredi 16 novembre}

\begin{document}
\maketitle

Le simulateur s'attend à recevoir, en argument, le nombre de cycles simulés,
ainsi que le circuit à simuler. Si le fichier s'appelle \verb|circuit.net|, et il
contient une variable \verb|var| qui est une rom, le simulateur lit le contenu de
la rom dans un fichier appelé \verb|circuit.var.rom|. Le contenu de ce fichier
doit contenir exactement le contenu de la rom, excepté pour l'espacement qui est
ignoré.\par
Pour une question de déterminisme (cf. les tests), l'initialisation de la ram et
des registres se fait à la valeur $0$. Toutes les variables en entrée du circuit
sont demandées sur l'entrée standard. Par défaut, le nom de la variable, et sa
taille, sont spécifiée. Cela peut être désactivé avec l'option \verb|--batch|,
ce qui est utile pour les tests. Les variables de sortie du circuit sont
affichées, précédées de leur nom, et éventuellement entre chevrons si la variable
est un bus.\par
Deux environnements sont maintenus à tout instant: celui des variables définies
au cycle précédent (dit \verb|old|), et celui des variables définies au cycle
actuel (dit \verb|current|). À la fin d'un cycle, les deux environnements sont
inversés. Les variables utilisées comme argument \verb|we|, \verb|wa| et
\verb|data| sont lues dans l'environnement \verb|old|, ce qui assure que
l'instruction ram n'induit pas de cycles topologiques pour ces arguments. De
même, les variables lues dans les registres sont lues dans l'environnement
\verb|old|. L'écriture des variables, et les autres lectures, sont effectuées
dans l'environnement \verb|current|.\par
À noter, ce fonctionnement correspond à la spécification de l'ordre de lecture
et d'écriture pour la ram, mais introduit un décalage d'un cycle supplémentaire
par rapport à que l'on pourrait intuitivement penser dans l'écriture.\par
Le code peut être testé avec \verb|make tests|. Il sera, entre autre, testé sur
les circuits fournis, et sur d'autres. Pour chacun, si le circuit prend a des
variables en entrée, plusieurs tests unitaires sont fournis, testant de façon
(plus ou moins) exhaustive les entrées possibles. Enfin, des tests reposant sur
des fonctionnalités qui ne sont pas dans la spécification sont fournis. Ils
mettent en évidence le comportement contre intuitif de l'implémentation de la
ram.\par
Le binaire peut être produit avec \verb|make|. Il sera alors en
\verb|out/netlist_simulator|. La documentation (que vous être en train de lire)
peut être produite avec \verb|make doc|. Une archive du projet peut être produite
avec \verb|make archive|. Les binaires intermédiaires du projet peuvent être
produit avec \verb|make build| (ce qui produit \verb|out/netlist_simulator|,
\verb|out/graph_test| et \verb|out/scheduler_test|). Ce projet est sur github, à
l'adresse\\
\url{https://github.com/theblackbeans/netlist}.
\end{document}
